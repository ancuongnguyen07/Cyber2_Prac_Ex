\section{Task 1 --- Becoming a Certificate Authority (CA)}
%

\textbf{What part of the certificate indicates this is a CA's certificate?}\\
%
When a CA issues a certificate, it includes the "Basic Constraints"
extension in the certificate and sets the "CA" field to "true". This
indicates that the certificate is a CA certificate and can be used to
issue other certificates.

\begin{verbatim}
    ...
    X509v3 Basic Constraints: critical
    CA:TRUE # this part indicates that it is a CA's cert
    ...
\end{verbatim}

\textbf{What part of the certificate indicates this a self-signed
certificate?}\\
%
The field of {\fontfamily{qcr}\selectfont Issuer} and {\fontfamily{qcr}\selectfont
Subject} matches each other, indicating this is a self-signed certificate.
In this case, \emph{www.modelCA.com} is both issuer and subject.

\textbf{Identify the values of elements in the RSA algorithm}\\
%
\begin{itemize}
    \item Public exponent \(e = 65537 (0x10001)\)
    \item Private exponent \(d = 207f433ab50870ad9f60ef862c97c3 \cdots
    79e0efcfadd5cc73c078bfbeb30401\)
    \item Modulus \(n = 00ceda7f8d3bf36960137521cafe28 \cdots
    2e79e89b1acb9acc1652418565e3c5ed\)
    \item Prime 1 \(p = 00f87e5650cf975dec0a08e0949af9 \cdots
    97013b7b65fc3b25a98f35166d2da1\)
    \item Prime 2 \(q = 00d51a26049cd54f64538dcd82305b \cdots
    c1804b846745fe4da0a0cdd674bccd\)
\end{itemize}

A full detail list of RSA elements are shown in \autoref{lst:rsa_element}.